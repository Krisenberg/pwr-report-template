\documentclass{pwrReport}


% Set document metadata
\tableOfContentsName{My Custom Contents}
\referencesName{Ź - r - ó - d - ł - a}
\figureName{Fig...}
\tableName{Tabelka}
\titlePageLogoLeft{images/logo_PWR_podpis.png}
\titlePageLogoRight{images/logo_WIT.png}

\courseName{My course}
\title{My title}
\author{Jan Kowalski}
\date{28.03.2025}

\begin{document}

\maketitle
\tableofcontents

\newpage

\section{Lorem}
    \subsection{Ipsum}
        \subsubsection{Other section import}
            \input{section_one}

\section{Text styling}
    This text is \textit{italics}, this \underline{underlined}, this \textbf{bolded}, and this
    \textbf{\textit{\underline{has everything}}!}


\section{Environments}
    \subsection{Lists}
        Itemized list:
        \begin{itemize}
            \item this
            \item is
            \item an itemized list.
        \end{itemize}

        Enumerated list:
        \begin{enumerate}
            \item this
            \item is
            \item an enumerated list.
        \end{enumerate}

    \subsection{Equations}
        Inline equation: $\delta=b^2-4ac$.

        \noindent Multi-line equations:
        \begin{equation}
            \sum_{n=1}^{5} n = 1+2+3+4+5=15
            \label{sum_equation}
        \end{equation}
        \noindent We can reference them later, e.g. equation \ref{sum_equation}.

    \subsection{Tables}
        Table example:
        \begin{table}[h]
            \centering
            \begin{tabular}{|l|l|l|}
                \hline
                a & b & c \\ \hline
                d & e & f \\ \hline
                g & h & i \\ \hline
            \end{tabular}
            \caption{Simple table example}
        \end{table}

    \newpage
    \subsection{Images}
        WUST logo inserted below.
        \begin{figure}[!htbp]%
            \centering
            \includegraphics[width=0.5\textwidth]{images/logo_PWR_podpis.png}
            \caption{Wroclaw University of Science and Technology logo}%
            \label{fig:wust_logo}
        \end{figure}

    \section{Code}
    \code{Python}
\begin{lstlisting}
@staticmethod
def analyze_behavior_patterns(
    logs: list[dict], allowed_ids: list[str], logger: structlog.BoundLogger
) -> tuple[list[str], list[str]]:
    ...
    positive_tracks, negative_tracks = set(), set()

    for log in logs:
        if log["track_id"] in allowed_ids:
            if log["event"] == "NEXT" and float(log["percent"]) < 30:
                negative_tracks.add(log["track_id"])
            elif log["event"] == "PREVIOUS" and float(log["percent"]) > 70:
                positive_tracks.add(log["track_id"])
            elif float(log["percent"]) >= 90:
                positive_tracks.add(log["track_id"])

    return list(positive_tracks), list(negative_tracks)
\end{lstlisting}

    
\bibliographystyle{plain}
\bibliography{references}
\end{document}

